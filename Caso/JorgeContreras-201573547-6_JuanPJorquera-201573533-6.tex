\documentclass[11pt,letterpaper]{article}
\usepackage[top=2.0cm, bottom=3cm, left=2.0cm, right=2.0cm]{geometry}
\usepackage[utf8]{inputenc}
\usepackage[T1]{fontenc}
\usepackage[spanish]{varioref}
\usepackage[activeacute, spanish, es-tabla]{babel}
\usepackage{fancyhdr}
\usepackage{multicol}
\usepackage{float}
\usepackage{textcomp}
\usepackage{ae,aecompl}
\usepackage{amssymb,amsmath}
\usepackage[pdftex]{graphicx}
\pagestyle{fancy} 
\pagenumbering{gobble}
\renewcommand{\headrulewidth}{0pt} 
\setlength{\headsep}{20pt} 
\setlength{\headheight}{65pt} 
\setlength{\textheight}{600pt} 
\setlength{\columnsep}{15pt} 
\newcommand{\universidad}{\small{Universidad Técnica Federico Santa María}}
\newcommand{\campus}{\small{Jorge Contreras 201573547-6}}
\newcommand{\semestre}{\small{Juan Pablo Jorquera  201573533-6}}

% Definiciones de Título e Integrantes de Experiencia
\newcommand{\titulo}{Informe Caso - Tarea 2 BD}
\newcommand{\integrantes}{\begin{tabular}{c}

\end{tabular}}

\renewcommand{\maketitle}
{
\thispagestyle{fancy}
\begin{center}
\begin{Large}
\textbf{\titulo}\\
\end{Large}
\end{center}
\vspace{0.3cm}
}


%ENCABEZADO

\fancyhead[R]{\begin{minipage}[b]{0.405\textwidth}
\begin{center}
\universidad \\ 
\campus \\ 
\lab \\ 
\semestre
\end{center}
\end{minipage}}
\fancyhead[L]{\vspace{15pt}\includegraphics[height=1.6cm]{Escudo.png}}
%%%%%%%%%%%%%%%%%%%%%%%%%%%%%%%%%%%%%%%%%%%%%%%%
%                                              %
% AQUI TERMINAN LAS DEFINICIONES DE ENCABEZADO %
% Y EMPIEZA EL CUERPO DEL DOCUMENTO            %
%                                              %
%%%%%%%%%%%%%%%%%%%%%%%%%%%%%%%%%%%%%%%%%%%%%%%%

\begin{document}
\maketitle
\section{Modelo Conceptual}
\vspace{0.6cm}
\begin{center}
	\includegraphics[scale=0.5]{conceptual.png}
\end{center}
\let\thefootnote\relax\footnote{En varios caso se asumieron cosas en las cardinalidades como:
	\begin{itemize}
		\item Cine puede tener 0 salas, ya que puede ser nuevo sin tenerlas asignadas aún.
		\item Sala puede tener 0 películas, ya que se puede encontrar en mantención.
	\end{itemize}
}
\footnote{Se considera que los asientos se van llenando de forma ordenada, de modo que el número de asiento me permite identificar el último posible.}
\footnote{}

\newpage
\section{Modelo Lógico}
\vspace{0.6cm}
\begin{center}
	\includegraphics[scale=0.5]{logico.png}
\end{center}
\footnote{En actores y directores se tomó el nombre como PK, ya que al momento de que el empleado ingrese uno nuevo el, no hay forma de determinar si estaba ya ingresado más que con dicho nombre.}

\end{document}
